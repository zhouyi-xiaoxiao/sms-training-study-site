\documentclass[UTF8,12pt,openany]{ctexbook}
\usepackage{../style/sms_training_style}

\begin{document}
\hypersetup{
  pdftitle={企业短信培训全知识点总表 V2.0},
  pdfauthor={整理:Codex},
  pdfsubject={规则基线:SMS-CN-RULE-v2026.02 / SMS-INTL-RULE-v2026.02 / SMS-OPS-RULE-v2026.02}
}

\begin{titlepage}
  \centering
  {\fontsize{30}{36}\selectfont\bfseries\color{themeblue}企业短信培训全知识点总表}\par
  \vspace{1.2em}
  {\Large 覆盖课程原文全部关键知识点}\par
  \vspace{1.6em}
  \begin{keybox}
  \mystrong{适用对象}\par
  销售、运营、交付、技术支持、项目管理。\par
  \vspace{0.3em}
  \mystrong{使用方法}\par
  先读第1章“全景地图”,再按岗位阅读对应章节,最后用第13章做上线前核对。
  \end{keybox}
  \vfill
  \coverline{版本:V2.0(出版级精修版)}
  \coverline{规则基线:SMS-CN-RULE-v2026.02 / SMS-INTL-RULE-v2026.02 / SMS-OPS-RULE-v2026.02}
  \coverline{匿名等级:Release-L2}
  \coverline{整理时间:2026年2月8日}
\end{titlepage}

\tableofcontents
\clearpage

\chapter{知识全景地图}

\section{一张图看懂企业短信}
\begin{enumerate}
  \item 准入:SP证、码号证、运营商落地。
  \item 发送对象:会员/用户,且遵循隐私同意与营销合规。
  \item 发送载体:主码号+子端口+签名+正文+引流信息。
  \item 发送链路:客户系统$\rightarrow$短信平台$\rightarrow$运营商/供应商$\rightarrow$终端。
  \item 状态闭环:提交回执、状态回执、上行回执、对账回执。
  \item 运营核心:成功率、时效、投诉、成本、稳定性。
\end{enumerate}

\section{课程核心结论}
\begin{keybox}
\begin{itemize}
  \item 企业短信不是“能发就行”,是“合规+触达+可运营”的系统工程。
  \item 大客户成功靠“规则前置+接入治理+持续运营”,而不是一次性交付。
  \item 客户问题80\%可归因于四类:号码质量、内容合规、通道策略、回执口径。
\end{itemize}
\end{keybox}

\chapter{出版级口径控制(本版新增)}

\section{术语统一清单}
\begin{longtable}{P{0.2\textwidth}P{0.24\textwidth}P{0.48\textwidth}}
\toprule
\mystrong{术语} & \mystrong{统一写法} & \mystrong{统一规则} \\
\midrule
子端口 & 子端口(SubID) & 禁用“扩展码/后缀码”作为主写法,仅可在注释中出现 \\
状态回执 & 状态回执(成功/失败) & “未知”仅表示暂未返回,不能写为最终状态 \\
携号转网 & 携号转网 & 禁用“企业号转码”作为正文主术语 \\
会员营销短信 & 会员营销短信 & 禁用“会销”作为发布版主术语 \\
引流信息 & 引流信息(链接/号码) & 必须明确“链接与号码均需报备” \\
\bottomrule
\end{longtable}

\section{客户匿名策略}
\begin{itemize}
  \item 发布版统一使用“案例编号+行业标签”命名,不使用客户实名。
  \item 如需保留背景强度,保留“头部/区域/国家级”等级描述,不保留可逆识别信息。
  \item 内部映射关系仅保留在受控文件,不进入公开学习资料。
\end{itemize}

\section{规则版本号}
\begin{longtable}{P{0.3\textwidth}P{0.24\textwidth}P{0.32\textwidth}}
\toprule
\mystrong{规则包} & \mystrong{版本} & \mystrong{适用范围} \\
\midrule
中国短信规则基线 & SMS-CN-RULE-v2026.02 & 签名、计费、回执、营销限制、投诉 \\
国际短信规则基线 & SMS-INTL-RULE-v2026.02 & Sender ID、报备、回填率、国家策略 \\
运营交付规则基线 & SMS-OPS-RULE-v2026.02 & 接入、压测、监控、重保、对账 \\
\bottomrule
\end{longtable}

\section{规则适用声明}
本知识点总表中的规则口径,统一适用版本基线:
SMS-CN-RULE-v2026.02、SMS-INTL-RULE-v2026.02、SMS-OPS-RULE-v2026.02。

\chapter{监管与准入知识点}

\section{资质体系}
\begin{longtable}{P{0.2\textwidth}P{0.74\textwidth}}
\toprule
\mystrong{资质} & \mystrong{知识点} \\
\midrule
增值电信业务经营许可证(SP证) & 企业开展短信等增值电信业务前提,含有效期与续期要求 \\
码号证 & 码号资源使用资格,不等于可直接发送 \\
基础电信业务经营许可证 & 主要由运营商及少数基础电信主体持有,部分招标会作为门槛 \\
\bottomrule
\end{longtable}

\section{码号落地}
\begin{itemize}
  \item 码号证获取后需在运营商落地,形成可用通道。
  \item 三网分离原则:移动/联通/电信分别落地、分别发送。
  \item 三网合一:同一发件标识在三网一致可见,保障更高,成本更高。
\end{itemize}

\section{监管趋势}
\begin{itemize}
  \item 从“先发后管”转向“先报备后发送”。
  \item 签名报备、引流信息报备成为前置条件。
  \item 营销短信退订文案统一规范化,减少模糊口径。
\end{itemize}

\chapter{码号、子端口、签名知识点}

\section{码号结构}
\begin{itemize}
  \item 常见为106开头。
  \item 前8位为基础码号段,后缀为可扩展子端口(SubID)。
  \item 总长度上限20位(课程口径)。
\end{itemize}

\section{大客户常见码号需求}
\begin{enumerate}
  \item 固定结尾(例如客服短号映射)。
  \item 总长度上限(如不超过11位/12位)。
  \item 三网一致可见(品牌统一展示)。
\end{enumerate}

\section{签名规则}
\begin{itemize}
  \item 国内短信签名格式固定:\texttt{【签名】}。
  \item 可用签名:企业全称、合规简称、已核准商标、部分可核验备案主体(按运营商规则)。
  \item 简称需唯一且不可跳字。
  \item 同一短信里除正式签名外,不应再出现方头括号,避免多签名判定。
\end{itemize}

\section{签名与子端口映射}
\begin{keybox}
\mystrong{必须牢记}\par
\begin{itemize}
  \item 一个子端口只能对应一个签名。
  \item 一个签名可以对应多个子端口。
  \item 子端口报备后,引流信息与签名关系也会被绑定管理。
\end{itemize}
\end{keybox}

\chapter{短信内容、分类、场景知识点}

\section{三大短信类型}
\begin{longtable}{P{0.14\textwidth}P{0.24\textwidth}P{0.22\textwidth}P{0.22\textwidth}}
\toprule
\mystrong{类型} & \mystrong{典型内容} & \mystrong{核心指标} & \mystrong{常见风险} \\
\midrule
验证码 & 登录、注册、找回密码 & 时效(秒级) & 轰炸、重发频控 \\
通知 & 动账、物流、系统提醒、订单状态 & 稳定触达 & 内容合规、误触达 \\
会员营销 & 活动促销、复购提醒、会员召回 & 触达+转化 & 投诉、退订、隐私合规 \\
\bottomrule
\end{longtable}

\section{营销短信底线}
\begin{itemize}
  \item 只能做会员营销,不做陌生人营销。
  \item 必须有退订口径:\texttt{拒收请回复R}。
  \item 发送时段受限,通常早8晚10,高危行业更严格。
\end{itemize}

\section{行业场景地图}
\begin{itemize}
  \item 电商:验证码+订单通知+大促营销。
  \item 物流:订单与配送通知为主。
  \item 银行保险:动账通知、验证、活动通知。
  \item 能源电力:缴费提醒、欠费通知、工单通知。
  \item 航旅出行:订单、延误、值机、升舱活动。
  \item 教育:上课提醒、课程通知、活动营销。
  \item 政务:通知与身份验证为主,安全合规要求高。
\end{itemize}

\chapter{计费与结算知识点}

\section{计费字符规则}
\begin{itemize}
  \item $\leq 67$字:1条。
  \item $>67$字:按67字分片计费。
  \item 140字$\Rightarrow$3条(非2条)。
  \item 签名、括号、标点、空格、链接都计费。
\end{itemize}

\section{常见计费模式}
\begin{enumerate}
  \item 成功计费:仅成功计费。
  \item 失败不计费:成功+未知计费(按平台与合同定义)。
  \item 提交计费:提交即计费(通常对平台收益更高)。
\end{enumerate}

\section{失败返还}
\begin{itemize}
  \item 常见于预付费客户。
  \item 提交时先预扣,72小时后按失败状态返还额度。
  \item 未知状态窗口会影响短期账面波动。
\end{itemize}

\section{长短信对账风险点}
\begin{riskbox}
长短信分片、补发、客户“只收一条状态”需求叠加时,最容易出现双方账单口径差异。
必须在合同或对账规则中提前约定:统计口径、容差范围、争议处理方式。
\end{riskbox}

\chapter{下发链路与回执知识点}

\section{链路节点}
客户触发$\rightarrow$客户平台$\rightarrow$接口提交$\rightarrow$短信平台处理$\rightarrow$运营商/供应商$\rightarrow$终端$\rightarrow$回执回传。

\section{回执三件套}
\begin{itemize}
  \item 提交回执:平台已接收请求。
  \item 状态回执:成功或失败(最终状态)。
  \item 上行回执:用户回复内容(R、数字口令、普通文本)。
\end{itemize}

\section{未知状态认知}
\begin{itemize}
  \item 未知是“暂未返回状态”,不是最终状态分类。
  \item 通常72小时内继续收敛为成功或失败。
  \item 正常未知率不应过高,异常增高需排查链路故障或号码质量。
\end{itemize}

\section{状态回传策略}
\begin{itemize}
  \item 可实时回推,也可按客户能力限流回推。
  \item 少数重点客户可开放主动拉取。
  \item 主动拉取要考虑资源占用、安全、隔离策略。
\end{itemize}

\chapter{风控、审核、投诉知识点}

\section{关键词机制}
\begin{itemize}
  \item 平台关键词库用于拦截违法违规内容。
  \item 关键词分组、分级,可按账号策略差异化配置。
  \item 语义可解释场景可做白名单化放通(合规前提下)。
\end{itemize}

\section{黑白名单机制}
\begin{itemize}
  \item 黑名单:强拦截,保护通道健康与投诉指标。
  \item 白名单:测试号/告警号/重保号放通,并可配优质专属资源。
  \item 黑名单解除需看级别、内容类型、证据链,不可“一刀切可解”。
\end{itemize}

\section{审核策略}
\begin{itemize}
  \item 大客户、低风险业务常免审或弱审。
  \item 小客户、高风险营销常需人工审核。
  \item 验证码通常不宜走重人工审核,避免时效损失。
\end{itemize}

\section{投诉治理}
\begin{itemize}
  \item 常见投诉入口:12321、运营商客服、通管局、12315等。
  \item 通道指标控制:百投比+绝对值双约束。
  \item 申诉材料:会员证明、授权链路、隐私协议同意证据等。
\end{itemize}

\chapter{接口与平台能力知识点}

\section{接口协议}
\begin{longtable}{p{0.2\textwidth}p{0.3\textwidth}p{0.42\textwidth}}
\toprule
\mystrong{协议} & \mystrong{适用场景} & \mystrong{关键点} \\
\midrule
HTTP/HTTPS & 绝大多数企业客户 & 需按接口文档开发;HTTPS适用于更高安全要求 \\
CMPP & 国内通信行业标准接口 & 行业内客户接入快,兼容性高 \\
SMPP & 国际短信常见标准 & 出海场景、国际供应链对接常用 \\
\bottomrule
\end{longtable}

\section{平台功能能力点}
\begin{itemize}
  \item 账号管理:开通、鉴权、权限控制。
  \item 资源池调度:多通道分流、权重策略、失败补发。
  \item 监控告警:成功率、时延、余额、通道健康、投诉指标。
  \item 统计报表:成功/失败/未知、点击、解析、UV/PV(按产品能力)。
  \item 安全与隔离:客户级隔离、接口限流、拉取保护。
\end{itemize}

\chapter{产品矩阵知识点(文本/富媒体/阅信/5G/语音/闪信/USSD/二进制短信)}

\section{文本短信}
优势:覆盖广、链路成熟、成本低。\par
限制:展示单一、交互弱。

\section{富媒体短信}
优势:图文/视频展示强、营销吸引力高。\par
限制:成本高于文本,模板审核与素材准备成本高。

\section{阅信(智能解析)}
优势:可卡片化展示、可按钮跳转、可做点击追踪。\par
限制:终端支持不一致,iOS链路更长,解析与短信双重成本。

\section{5G消息}
优势:交互丰富、可Chatbot。\par
限制:终端覆盖与可寻址规模仍是现实约束。

\section{语音短信/语音验证码}
优势:作为文本验证码补充,提高可达性。\par
限制:成本与用户接听行为影响较大。

\section{闪信}
优势:强提醒(来电前提示等)。\par
限制:机型稳定性与展示时序不完全一致。

\section{USSD消息(会话型)}
优势:实时双向交互、弱网可用、终端覆盖广、无需App。\par
限制:文本菜单体验有限、会话超时后需重进、单次承载有限。

\section{二进制短信(Binary SMS)}
优势:可通过短信通道传输控制类小数据,适配M2M与设备管理。\par
限制:实现复杂、单条载荷小、终端兼容性需要专项联调。

\chapter{国际短信知识点}

\section{基础规则}
\begin{itemize}
  \item 国际短信以Sender ID识别品牌。
  \item 不同国家报备规则、模板规则、退订规则差异大。
  \item 部分国家报备存在注册费和月租费。
\end{itemize}

\section{关键指标}
\begin{itemize}
  \item 成功率:通知类常看。
  \item 回填率:验证码核心指标。
  \item 到达时延:验证码时效体验关键。
\end{itemize}

\section{WhatsApp补充通道}
\begin{itemize}
  \item 在主流国家可作为高触达互动通道。
  \item 需考虑模板审核、会话窗口、国家使用习惯。
\end{itemize}

\chapter{客户接入与商务知识点}

\section{接入全流程}
需求确认$\rightarrow$投标/商务$\rightarrow$账号开通$\rightarrow$接口联调$\rightarrow$报备$\rightarrow$测试$\rightarrow$上线$\rightarrow$运营$\rightarrow$对账续约。

\section{测试策略}
\begin{itemize}
  \item 点测:验证链路可用。
  \item 压测:验证平台和通道承载。
  \item 批测:真实业务小流量观察后扩量。
\end{itemize}

\section{压测必问清单}
\begin{enumerate}
  \item 目标QPS是多少。
  \item 只测平台接入还是测全链路。
  \item 压测时间窗、持续时长。
  \item 是否回推状态,回推速率要求。
  \item 是否会与线上高峰冲突。
\end{enumerate}

\section{客户分层策略}
\begin{itemize}
  \item 大中直客:高频沟通、重保策略、定制能力、报告化服务。
  \item 小微客户:自服务优先、预付费优先、标准化运营。
  \item 渠道客户:资源效率优先、质量与成本平衡。
\end{itemize}

\chapter{销售与运营协同知识点}

\section{销售必采集信息}
\begin{itemize}
  \item 行业、体量、短信类型占比。
  \item 当前供应商、痛点、替换诉求。
  \item 码号要求(固定尾号、总长度、三网合一)。
  \item 回执方式(回推/拉取/限流)。
  \item 历史投诉与百投比大致水平。
  \item 是否存在定制化功能需求。
\end{itemize}

\section{运营必建立机制}
\begin{itemize}
  \item 通道池策略与失败补发策略。
  \item 账号级风控参数(频控、关键词、黑白名单、地区策略)。
  \item 异常监控与告警分级。
  \item 问题闭环(定位、反馈、复盘、规则更新)。
\end{itemize}

\section{影响利润四因子}
\begin{enumerate}
  \item 单价。
  \item 计费口径。
  \item 通道要求复杂度。
  \item 最终成功率(与有效号码质量强相关)。
\end{enumerate}

\chapter{上线前与日常运营核对表}

\section{上线前核对(Checklist)}
\begin{keybox}
\begin{enumerate}
  \item 资质/合同/结算方式确认。
  \item 签名报备、引流报备状态确认。
  \item 通道资源池与备份策略确认。
  \item 回执策略与口径确认。
  \item 风控参数确认(频次、时间窗、黑白名单、关键词)。
  \item 压测或批测报告确认。
  \item 节点联系人与应急机制确认。
\end{enumerate}
\end{keybox}

\section{日常监控核心指标}
\begin{itemize}
  \item 成功率、失败率、未知率。
  \item 时延(提交到回执、提交到到达)。
  \item 投诉量、百投比。
  \item 账户余额、通道余额。
  \item 大客户回执堆积与拉取异常。
\end{itemize}

\section{异常排障优先级}
\begin{enumerate}
  \item 是否全量失败(接口鉴权/网络/通道故障)。
  \item 是否集中失败(某运营商/某省份/某模板)。
  \item 是否规则拦截(关键词、黑名单、频控、时间窗)。
  \item 是否号码质量问题(空号、停机、无信号、携转)。
  \item 是否回执口径问题(未知窗口、长短信分片、映射偏差)。
\end{enumerate}

\chapter{必背速记页(考前10分钟)}

\begin{tipbox}
\mystrong{速记1:状态}\par
最终状态只有成功和失败;未知是暂未返回状态,72小时内会收敛。
\end{tipbox}

\begin{tipbox}
\mystrong{速记2:计费}\par
67字一条;超67按67拆分;140字=3条;签名和符号都计费。
\end{tipbox}

\begin{tipbox}
\mystrong{速记3:映射}\par
子端口\,$\leftrightarrow$\,签名是“一对一(端口到签名)”;签名可对应多个子端口。
\end{tipbox}

\begin{tipbox}
\mystrong{速记4:营销}\par
必须会员前提;必须退订口径;必须遵守时间窗与频次限制。
\end{tipbox}

\begin{tipbox}
\mystrong{速记5:接入}\par
先问清需求再开账号;先做小流量验证再扩量;压测必须明确QPS和时间窗。
\end{tipbox}

\appendix
\chapter{修订说明与变更记录}
\section{修订声明}
\begin{riskbox}
本文档已完成可学习化修订,并已执行出版级精修:统一术语、统一客户匿名策略、统一规则版本号。
保留少量课堂表达,仅用于维持学习语境,不影响规则准确性。
\end{riskbox}

\section{版本变更记录}
\begin{longtable}{P{0.13\textwidth}P{0.16\textwidth}P{0.55\textwidth}}
\toprule
\mystrong{版本} & \mystrong{日期} & \mystrong{变更说明} \\
\midrule
V1.0 & 2026-02-08 & 初版:课程知识点结构化整理 \\
V2.0 & 2026-02-08 & 出版级精修:术语统一、匿名策略、规则版本化、附录补全 \\
V2.1 & 2026-02-09 & 整合“消息类型介绍”材料:新增USSD与二进制短信知识点并细化闪信 \\
\bottomrule
\end{longtable}

\section{A.4 对外发布前检查清单}
\begin{keybox}
\begin{enumerate}
  \item 术语是否全部符合“术语统一标准”。
  \item 客户信息是否全部达到 Release-L2 匿名等级。
  \item 规则口径是否全部标注版本号。
  \item 时间窗、计费、回执、频控描述是否与当前规则一致。
  \item 图表标题、单位、缩写(QPS、MO、MT)是否统一。
  \item 是否移除内部群名、个人姓名、私有项目代号。
  \item PDF 元信息与封面版本信息是否一致。
\end{enumerate}
\end{keybox}

\end{document}
