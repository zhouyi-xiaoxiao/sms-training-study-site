\documentclass[UTF8,12pt,openany]{ctexbook}
\usepackage{../style/sms_training_style}

\begin{document}
\hypersetup{
  pdftitle={企业短信培训学习手册(专业文稿版)V4.0},
  pdfauthor={整理:Codex},
  pdfsubject={规则基线:SMS-CN-RULE-v2026.02 / SMS-INTL-RULE-v2026.02 / SMS-OPS-RULE-v2026.02}
}

\begin{titlepage}
  \centering
  {\fontsize{30}{36}\selectfont\bfseries\color{themeblue}企业短信培训学习手册(专业文稿版)\par}
  \vspace{1.2em}
  {\Large 由课堂逐字内容重写为可学习、可发布的书稿体文档}\par
  \vspace{1.8em}
  \begin{keybox}
  \mystrong{文档定位}\par
  本文档基于完整培训语料进行专业文稿化重写,目标是把课堂口语内容沉淀为可长期复用的知识资产。
  重写过程坚持三项原则:
  \begin{itemize}
    \item 不改变业务结论,不削弱规则边界,不稀释实操经验。
    \item 去除对话体、口头语和重复表达,统一为书面叙事结构。
    \item 对关键名词、机制与口径进行系统化解释,保证可直接教学与复盘。
  \end{itemize}
  \end{keybox}
  \vfill
  \coverline{版本:V4.0(专业文稿版)}
  \coverline{规则基线:SMS-CN-RULE-v2026.02 / SMS-INTL-RULE-v2026.02 / SMS-OPS-RULE-v2026.02}
  \coverline{匿名等级:Release-L2(案例匿名化,行业标签保留)}
  \coverline{整理时间:2026年2月8日}
  \coverline{用途:企业短信业务学习、交付培训、销售与运营协同、项目复盘}
\end{titlepage}

\tableofcontents
\clearpage

\chapter{课程总览与学习路径}

\section{培训时间与组织方式}
本次课程按“2月启动、3月集中授课”的节奏组织。春节前后两周预留为空档期,3月开始按照固定计划连续授课,单次课程时段主要集中在17:00--18:30,采用线上与线下并行的方式。课程定位为企业短信业务基础能力培训,目标人群覆盖销售、运营、交付与技术支持岗位。

\section{四课主线}
\begin{longtable}{P{0.14\textwidth}P{0.78\textwidth}}
\toprule
\mystrong{课次} & \mystrong{核心主题} \\
\midrule
第1课 & 企业短信基础:定义、构成、签名与子端口、计费口径、状态回执、产品形态(文本/富媒体/阅信/5G) \\
第2课 & 监管与准入:SP证与码号证、码号落地、三网合一、黑白名单、上行下行、国际短信关键术语 \\
第3课 & 规则深化:接口协议、投诉治理、签名规范、发送限制、有效号码、计费与运营机制 \\
第4课 & 客户接入与交付:接入流程、测试方法(点测/压测/批测)、服务分层、典型案例复盘 \\
\bottomrule
\end{longtable}

\section{零基础先读(5分钟入门)}
若你首次接触企业短信,建议先记住以下四个“骨架概念”,再进入后续章节:
\begin{enumerate}
  \item \mystrong{业务本质}:企业短信是 B2C 触达能力,费用由企业承担。
  \item \mystrong{链路关键}:客户提交请求 $\rightarrow$ 平台风控处理 $\rightarrow$ 通道下发 $\rightarrow$ 回执返回。
  \item \mystrong{合规底线}:签名与引流需报备,营销必须有会员前提和退订口径。
  \item \mystrong{结果口径}:最终状态只有成功/失败;未知是暂未返回,不是第三终态。
\end{enumerate}
完成上述四点理解后,再学习“计费规则、风控策略、接入压测、客户交付”,理解速度会显著提升。

\section{出版级口径控制}
\subsection{术语统一标准}
\begin{longtable}{P{0.15\textwidth}P{0.22\textwidth}P{0.18\textwidth}P{0.33\textwidth}}
\toprule
\mystrong{标准术语} & \mystrong{统一写法} & \mystrong{课堂常见别称} & \mystrong{说明} \\
\midrule
主码号 & 主码号(前8位) & 码号、主端口 & 对外文稿统一写“主码号” \\
子端口 & 子端口(SubID) & 扩展、后缀、扩展码 & 对外文稿统一写“子端口(SubID)” \\
携号转网 & 携号转网 & 转网、号转码 & 统一写“携号转网” \\
会员营销短信 & 会员营销短信 & 会销短信 & 统一写“会员营销短信” \\
状态回执 & 状态回执(成功/失败) & 状态报告、回执 & “未知”不作为最终状态 \\
引流信息 & 引流信息(链接/号码) & 引流、落地链接 & 链接与号码均需纳入报备管理 \\
\bottomrule
\end{longtable}

\subsection{匿名策略与规则版本}
本稿采用 Release-L2 匿名等级:不公开客户实名、个人姓名、内部群名和私有项目代号;仅保留“行业+场景+能力”信息。全文规则口径统一适用以下版本基线:
\begin{itemize}
  \item SMS-CN-RULE-v2026.02(中国短信规则基线)
  \item SMS-INTL-RULE-v2026.02(国际短信规则基线)
  \item SMS-OPS-RULE-v2026.02(运营交付规则基线)
\end{itemize}

\chapter{企业短信的定义、价值与边界}

\section{定义与业务本质}
企业短信并非个人社交短信的延伸,而是企业对其用户提供通知、验证和会员营销服务的基础通信能力。其本质属于 B2C 信息触达服务:企业发起业务消息,平台负责合规处理与链路下发,终端用户接收消息,费用由企业承担而非终端用户承担。

\section{为什么企业短信仍是基础设施}
尽管个人沟通大量迁移至即时通讯工具,但企业业务链路仍高度依赖短信,原因包括:
\begin{enumerate}
  \item 实名身份绑定能力强,适配注册、登录和交易验证。
  \item 与业务系统天然耦合,适合自动触发型通知。
  \item 不依赖特定App生态,覆盖能力稳定。
\end{enumerate}

\section{合规边界}
短信服务不能脱离用户授权体系。尤其在营销场景中,必须以会员关系或明确授权为前提。若企业向非会员发送营销信息,会直接触发隐私、投诉与监管风险。因此,短信业务是“可触达能力”与“合规约束能力”的组合,不是单向度的批量发送能力。

\chapter{一条企业短信的完整构成}

\section{主码号与子端口}
企业短信通常由主码号与子端口共同形成发件标识。主码号前缀为固定资源,子端口用于业务映射、签名绑定与差异化识别。课程口径中,主码号前8位固定,整体位长可扩展至20位,子端口最大扩展长度可达12位。

\section{签名机制}
国内短信签名采用固定格式 \texttt{【签名】},且必须通过报备后才能用于下发。可用于报备的签名来源主要包括企业全称、合规简称、已核准商标及部分可核验备案主体(以运营商当期规则为准)。简称报备必须满足两个条件:不可跳字、具备唯一性。

\section{签名与子端口映射规则}
子端口到签名是“一对一”关系,即同一子端口仅可绑定一个签名;签名到子端口可为“一对多”,即一个签名可在多个子端口上报备。该规则是后续引流报备、投诉追溯与资源治理的基础。

\section{正文与引流信息}
正文必须与签名主体业务相关。营销短信尾部必须包含统一退订口径 \texttt{拒收请回复R}。正文中的引流信息(链接、电话号码)已纳入前置报备体系,不可随意替换。若同一主体存在多组链接/号码组合,通常需要增加子端口以满足一一映射管理。

\chapter{短信类型、业务场景与时效特征}

\section{验证码短信}
验证码短信核心指标是时效,通常要求秒级到达。用户行为与短信触发强绑定,若延迟过高,直接影响登录、注册和交易转化。

\section{通知短信}
通知短信覆盖动账通知、订单通知、物流通知、系统提醒、工单通知等场景。其时效要求低于验证码,但对稳定触达和状态可追踪要求较高。

\section{会员营销短信}
会员营销短信用于促活、复购、促销和召回。其典型特征是批量下发、QPS较高、时效容忍度相对更宽。其底线规则是“会员前提+退订口径+时间窗限制+频控限制”。

\section{行业场景映射}
\begin{itemize}
  \item 电商:验证码、订单通知、活动营销并存,大促期高峰明显。
  \item 物流:通知占主导,时效与稳定性要求高。
  \item 银行与保险:动账与安全验证占比高,投诉治理要求严。
  \item 能源与电力:缴费提醒、欠费通知、工单通知为核心。
  \item 航旅与出行:订单、延误、值机等通知对实时性要求高。
  \item 教育与内容平台:通知与营销并行,周期性触达明显。
\end{itemize}

\chapter{下发链路、回执机制与状态口径}

\section{标准链路}
一条短信从业务触发到结果闭环,通常经历以下路径:
\begin{enumerate}
  \item 客户业务系统发起短信请求。
  \item 通过 HTTP/HTTPS/CMPP 等接口提交至短信平台。
  \item 平台执行签名校验、报备校验、风控校验、频控校验等处理。
  \item 合规消息进入运营商或供应商通道,完成终端下发。
  \item 平台回传状态回执并处理上行消息。
\end{enumerate}

\section{提交回执与状态回执}
提交回执表示“平台已接收请求”;状态回执表示“发送结果”。状态回执最终口径仅有成功与失败两类。

\section{未知状态的正确理解}
“未知”不是第三种最终状态,而是“暂未返回最终状态”的中间态。通常在72小时窗口内逐步收敛为成功或失败。对账与复盘应以72小时后的稳定口径为准。

\section{回执推送与拉取策略}
平台支持按客户承载能力调整回执回推速率。少量重点客户可采用主动拉取模式,但需同步评估资源占用、访问安全、数据隔离与拉取纪律,避免状态堆积和平台侧拥塞。

\chapter{计费、分片与对账机制}

\section{字符计费规则}
\begin{keybox}
\mystrong{统一口径}\par
\begin{itemize}
  \item 短信长度 $\leq67$ 字:按1条计费。
  \item 长短信长度 $>67$ 字:按67字分片计费。
  \item 示例:140字按3条计费,而非2条。
\end{itemize}
\end{keybox}

\section{计费字符范围}
凡出现在短信内容中的字符均计费,包括签名、括号、标点、空格、链接、换行等,不存在“签名不计费”“空格不计费”的特例口径。

\section{计费模式与失败返还}
常见计费模式包括成功计费、失败不计费(按合同定义)与提交计费。预付费模式下,平台一般采用“先预扣、后返还”机制:失败条数在72小时窗口后返还。

\section{长短信分片与账单差异}
长短信可能出现分片级成功/失败并触发补发。若客户端采用“长短信单条状态口径”,而平台采用“分片口径”,双方账单会产生统计差异。该问题必须在合同或对账规则中提前约定。

\chapter{监管准入与码号治理}

\section{资质体系}
开展企业短信服务需具备增值电信业务经营许可证(SP证)与码号证。码号证仅代表号码资源资格,不代表可直接发送;只有完成运营商落地形成通道后,才能承载真实业务。

\section{三网落地与三网合一}
短信发送遵循分网原则:移动、联通、电信分别落地、分别发送。三网合一是指同一发件标识在三网一致可见,通常适用于对品牌一致性要求高的大型客户。由于资源稀缺与维护难度高,三网合一方案通常成本更高。

\section{投诉指标约束}
投诉治理并非售后补救动作,而是通道生存条件。通道治理通常同时考核投诉比率与投诉绝对值,超限可能触发限流、罚则或关停。销售与运营在接入前必须评估客户历史投诉水平与业务风险等级。

\chapter{平台风控与发送限制}

\section{关键词、黑白名单与人工审核}
平台通过关键词库、黑名单库与人工审核形成多层风控。黑名单并非单一等级,不同等级对应不同处置策略;白名单用于测试号、告警号或重保号放行。高风险营销业务通常采用更严格审核策略。

\section{时间窗与频控}
\begin{itemize}
  \item 通知与验证码:通常可全天发送(以合规场景为前提)。
  \item 会员营销:常规时间窗为早8晚10。
  \item 高风险营销:可进一步收紧到早8晚6等策略。
\end{itemize}
平台可对禁发时段提交的营销短信采取“直接失败”或“延时排队”两种策略。

\section{防轰炸策略}
验证码防轰炸用于限制同号码在短时窗内被多签名、多业务高频触发,防止恶意轰炸和终端骚扰。该策略本质是用户保护机制,不是单纯限流机制。

\section{地区与账号级策略}
针对高风险业务可启用地区屏蔽策略和更严频控策略;策略粒度可覆盖账号级、签名级、号码级,且可依据投诉与效果数据动态优化。

\chapter{产品矩阵:文本、富媒体、阅信、5G与扩展消息类型}

\section{文本短信}
文本短信是当前最稳定、最广覆盖、最低门槛的基础能力,适合验证码与通知主链路。

\section{富媒体短信}
富媒体短信支持图文、音视频组合,适用于营销展示与活动传播。其优点是信息表现力强,缺点是成本更高、素材和审核链路更复杂。

\section{阅信(智能解析短信)}
阅信通常以“文本+解析链接”方式提交,终端支持时可呈现卡片式展示,并提供按钮跳转到App、小程序、网页或客服电话。解析链路可提供聚合统计(如解析数、点击率、UV/PV),但明细粒度受产品与通道能力约束。iOS端通常需要额外点击进入解析页。

\section{5G消息}
5G消息具备更丰富的交互结构(多按钮、会话、Chatbot等),但规模化应用仍受终端覆盖与可寻址能力限制。

\section{语音短信与闪信}
语音验证码常作为文本验证码补充,用于提高可达性;闪信用于强提醒场景(如来电前提醒),但在不同终端上的展示稳定性存在差异。

\section{USSD消息(会话型菜单交互)}
USSD(Unstructured Supplementary Service Data)是基于GSM网络的实时会话协议,典型入口是“\texttt{*...\#}”代码。它与文本短信“存储转发”机制不同:USSD在会话期间保持在线交互,消息通常不落地到终端收件箱。
\begin{itemize}
  \item 典型场景:余额查询、话费充值、简易银行菜单、功能机交互服务。
  \item 主要优势:实时双向、弱网可用、终端覆盖广、无需安装App。
  \item 主要限制:文本菜单体验较弱、会话有超时窗口、单次承载字符有限。
\end{itemize}

\section{二进制短信(Binary SMS)}
二进制短信仍走短信通道,但负载是二进制数据而非文本字符。常通过 DCS/UDH 等头信息告诉终端“如何解析与处理”。
\begin{itemize}
  \item 典型场景:设备配置下发、M2M控制指令、WAP Push、SIM应用更新。
  \item 主要优势:依托短信网络,覆盖广、可在小数据控制场景中稳定送达。
  \item 主要限制:单条承载上限小(约140字节级)、实现与联调复杂、终端兼容性需要验证。
\end{itemize}

\section{闪信(Flash SMS,Class 0)细化说明}
闪信本质是 Class 0 短信:消息优先弹窗显示,通常不进入收件箱。它适合强时效提醒,不适合常规营销。
\begin{itemize}
  \item 典型场景:紧急告警、安全提醒、一次性验证码(需谨慎评估锁屏可见风险)。
  \item 主要优势:可见性高、触达后注意力强。
  \item 主要限制:侵入性强、可回看性弱;在锁屏场景下存在信息暴露风险。
\end{itemize}

\chapter{国际短信专项}

\section{核心标识与国家差异}
国际短信通常以 Sender ID 识别品牌,不同国家对 Sender ID 形态、报备资料、审批周期、退订口径要求不同。常见形态包括纯数字、纯字母或混合格式。

\section{关键指标}
国际通知场景常看成功率;国际验证码场景更关注回填率,即“收到验证码后实际填回业务页面的比例”。回填率比单纯提交成功率更能反映链路可用性与用户体验。

\section{DND与补充通道}
部分国家存在 DND(防骚扰)机制,DND命中号码可能不可触达。某些国家中,WhatsApp等通道可作为短信补充,但仍需遵守当地模板与会话规则。

\section{成本结构}
国际短信除单条发送成本外,部分国家可能存在 Sender ID 注册费、月租费等附加成本,需在商务阶段提前告知客户并纳入报价。

\chapter{客户接入、压测与上线治理}

\section{接入方式}
\subsection{Web自服务}
适用于小微客户或无技术团队客户。客户可在页面完成签名模板配置、号码导入、短信发送与结果查询。

\subsection{API接口接入}
适用于中大型客户。常见协议包括 HTTP/HTTPS、CMPP,国际场景常用 SMPP。此类客户通常具备多供应商调度能力,对接口稳定性和回执治理要求更高。

\section{标准接入流程}
需求澄清 $\rightarrow$ 账号开通 $\rightarrow$ 联调测试 $\rightarrow$ 报备完成 $\rightarrow$ 测试验收 $\rightarrow$ 正式上线 $\rightarrow$ 运营复盘。

\section{三类测试方法}
\begin{enumerate}
  \item 点测:验证接口与基础链路可用。
  \item 压测:验证平台承载与全链路吞吐。
  \item 批测:切入小规模真实业务,观察持续稳定性后再扩量。
\end{enumerate}

\section{压测必问清单}
压测前必须确认五个参数:目标QPS、压测模式(仅平台/全链路)、开始时间、持续时长、回执策略。高QPS压测必须提前联动运营与技术,避免冲击在线业务。

\section{上线前核对}
上线前需完成:签名报备与引流报备状态确认、资源池策略确认、回执策略确认、风控参数确认、应急联系人确认与故障升级路径确认。

\chapter{服务分层与交付策略}

\section{大中直客}
大中直客通常具有高体量、高SLA、高安全要求特征。服务策略应采用高频沟通、重保机制、定制能力和报告化交付。

\section{小微客户}
小微客户重点在于快速接入和稳定可用。推荐“标准化流程+自服务+预付优先”的低摩擦交付模式,避免高人工成本吞噬利润。

\section{渠道客户}
渠道合作核心是资源能力协同。评估维度应聚焦成本、质量、稳定性与交付速度,而非单一价格。

\section{运营闭环}
客户服务应形成“监控发现--定位分析--策略调整--结果回看”的闭环机制。大促、节假日、政策切换期应执行专项重保和值班制度。

\chapter{匿名案例复盘}

\section{案例A:区域能源集团}
该类客户以缴费提醒、欠费通知、工单通知为主,合作周期长,强调稳定性、品牌一致性与合规可审计。

\section{案例B:头部电商生态客户}
该类客户在大促期高峰显著,QPS要求高,需重点保障平台承载、通道稳定和状态回传能力。

\section{案例C:国家级身份认证项目}
该类项目安全性和审计要求极高,重点在于接口安全、状态准确、可追溯和持续可用。

\section{案例D:头部互联网平台定制项目}
该类项目体现“平台能力输出”而非单通道售卖,核心价值在于技术能力、规则体系与持续运维交付。

\section{案例E:三甲医院私有化项目}
该类项目强调私有化部署、长期运维与可控治理,通常具有较高客户粘性和复购潜力。

\section{案例F:快消品牌短链追踪项目}
该类项目通过短信短链实现二次触达追踪,关注 UV/PV、点击链路和转化归因能力。

\chapter{名词解释与机制详解(教学词典)}

\begin{longtable}{P{0.16\textwidth}P{0.23\textwidth}P{0.21\textwidth}P{0.26\textwidth}}
\toprule
\mystrong{名词} & \mystrong{定义} & \mystrong{实操要点} & \mystrong{常见误区} \\
\midrule
SP证 & 增值电信业务经营许可证,短信业务准入资质之一 & 到期前续期,投标时常作为门槛资质 & 误以为有SP证即可直接发短信 \\
码号证 & 企业可申请并持有的码号资源资格证明 & 获取后需运营商落地形成通道 & 误以为码号证=可发送能力 \\
落地 & 在运营商侧完成码号可用化配置的过程 & 分网落地、分网治理 & 误以为一次落地可发三网 \\
三网合一 & 同一发件标识在移动/联通/电信一致可见 & 常用于品牌一致性要求高场景 & 误以为三网合一成本不变 \\
主码号 & 发件标识的基础段(前8位) & 与子端口共同组成完整标识 & 与子端口混用概念 \\
子端口(SubID) & 主码号后缀,用于签名和业务映射 & 一端口一签名;一签名可多端口 & 误以为一端口可绑定多签名 \\
签名报备 & 将签名提交运营链路审核备案 & 未报备不可发送 & 误以为签名仅内部配置即可 \\
引流信息报备 & 对正文中的链接和号码进行前置报备 & 链接变更需同步更新报备 & 忽略号码也属于引流信息 \\
MT & 下行短信(平台发给用户) & 用于统计发送侧能力 & 与MO混淆 \\
MO & 上行短信(用户回复平台) & 退订、口令回复依赖MO链路 & 误以为所有链路都天然支持MO \\
提交回执 & 平台确认“已接收请求” & 用于判断接口可用性 & 误以为提交成功=发送成功 \\
状态回执 & 发送结果回执(成功/失败) & 用于对账、结算、复盘 & 将未知当作第三终态 \\
未知状态 & 暂未返回最终状态的中间态 & 72小时窗口后再做稳定口径对账 & 将未知直接等同失败 \\
QPS & 每秒处理请求或发送条数能力指标 & 压测、重保、容量规划核心指标 & 只关注峰值,不关注持续时长 \\
关键词策略 & 基于内容词库的合规拦截机制 & 分级配置,按账号风险管理 & 误以为关键词永远不可放行 \\
黑名单 & 不可触达或高风险号码集合 & 分级治理,非全部可解除 & 误以为所有黑名单可人工解封 \\
白名单 & 特殊放行号码集合 & 常用于测试号、告警号、重保号 & 误以为白名单可无限制发送 \\
防轰炸策略 & 防止同号码短时高频被验证码冲击 & 控制时间窗与频率阈值 & 误以为仅是成本控制策略 \\
时间窗控制 & 限制特定类型短信发送时段 & 营销场景必须严格执行 & 忽略时区与业务特殊窗口 \\
频控 & 限制单号码、单账号发送频次 & 按风险等级动态调整 & 仅看成功率忽视投诉风险 \\
失败返还 & 预付费模式下失败条数返还机制 & 72小时后返还更稳定 & 当日即要求绝对精确返还 \\
长短信分片 & 超67字后按67字分片发送与计费 & 对账必须统一分片口径 & 误以为140字计2条 \\
Sender ID & 国际短信发件标识 & 按国家规则申请与维护 & 各国规则想当然通用 \\
DND & 防骚扰名单,命中后可能不可触达 & 需在国家规则内规避触发 & 把DND当作临时网络问题 \\
SMPP & 国际短信常用标准接口协议 & 出海系统对接常见 & 与国内CMPP混同 \\
CMPP & 国内运营商体系常用标准协议 & 行业内系统接入速度快 & 误以为所有客户都适用 \\
\makecell[l]{HTTP/\\HTTPS} & 通用接口协议,HTTPS含传输加密 & 大多数企业客户首选 & 误以为HTTP不需任何安全治理 \\
USSD & GSM会话型菜单交互协议(常见\texttt{*...\#}) & 实时交互、弱网可用、消息通常不入箱 & 与SMS存储转发机制混淆 \\
Binary SMS & 负载为二进制数据的短信形态 & 适合M2M控制与配置下发 & 误以为可无限承载数据 \\
Flash SMS & Class 0 短信,优先弹窗显示 & 强提醒、高可见;通常不入箱 & 用于常规营销引发强干扰 \\
回填率 & 验证码被用户实际填回比例 & 国际验证码场景核心指标 & 用成功率替代回填率评估 \\
私有化部署 & 在客户侧专属部署短信平台能力 & 粘性高、运维要求高 & 误以为私有化是一次性交付 \\
\bottomrule
\end{longtable}

\appendix
\chapter{缩写速查}
\begin{longtable}{P{0.16\textwidth}P{0.76\textwidth}}
\toprule
\mystrong{缩写} & \mystrong{释义} \\
\midrule
QPS & 每秒请求/处理条数(Queries Per Second) \\
MT & 下行短信(Mobile Terminated) \\
MO & 上行短信(Mobile Originated) \\
SP & 增值电信业务服务提供者资质体系中的通用称谓 \\
SubID & 子端口编号 \\
SLA & 服务等级协议(Service Level Agreement) \\
DND & 防骚扰机制(Do Not Disturb) \\
\bottomrule
\end{longtable}

\chapter{A.4 对外发布前检查清单}
\begin{keybox}
\begin{enumerate}
  \item 术语是否全部符合“术语统一标准”。
  \item 客户信息是否全部达到 Release-L2 匿名等级。
  \item 规则口径是否全部标注版本号。
  \item 时间窗、计费、回执、频控描述是否与当前规则一致。
  \item 图表标题、单位、缩写(QPS、MO、MT)是否统一。
  \item 是否移除内部群名、个人姓名、私有项目代号。
  \item PDF 元信息与封面版本信息是否一致。
\end{enumerate}
\end{keybox}

\chapter{修订说明与版本记录}
\begin{riskbox}
本文档已完成可学习化修订与专业文稿化重写,并执行出版级精修:统一术语、统一客户匿名策略、统一规则版本号。
本版已经从课堂对话体重构为书稿体;逐字稿保留在单独文档,不与本稿混用。
\end{riskbox}

\begin{longtable}{P{0.13\textwidth}P{0.16\textwidth}P{0.55\textwidth}}
\toprule
\mystrong{版本} & \mystrong{日期} & \mystrong{变更说明} \\
\midrule
V1.0 & 2026-02-08 & 完成可学习化修订与补全,形成学习版文档 \\
V2.0 & 2026-02-08 & 完成出版级精修:术语统一、匿名策略、规则版本化 \\
V3.0 & 2026-02-08 & 完成书稿体重写:去对话化、专业叙述化、名词解释体系化 \\
V4.0 & 2026-02-08 & 命名升级为“学习手册”;新增零基础导读,强化初学者可读性 \\
V4.1 & 2026-02-09 & 整合“消息类型介绍”材料:新增USSD、二进制短信、闪信细化章节与术语 \\
\bottomrule
\end{longtable}

\end{document}
