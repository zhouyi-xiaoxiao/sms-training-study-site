\documentclass[UTF8,12pt,openany]{ctexbook}
\usepackage{../style/sms_training_style}
\usepackage{paracol}
\newcommand{\ansbadge}[1]{\fcolorbox{themeblue}{themeblue!10}{\strut\textbf{\;#1\;}}}
\newcommand{\ansline}[1]{\textbf{答案:}\ansbadge{#1}}
\newcommand{\expline}[1]{\textbf{解释:}#1}

\begin{document}
\hypersetup{
  pdftitle={企业短信知识题库(学习测评版)V3.0},
  pdfauthor={整理:Codex},
  pdfsubject={规则基线:SMS-CN-RULE-v2026.02 / SMS-INTL-RULE-v2026.02 / SMS-OPS-RULE-v2026.02}
}

\begin{titlepage}
  \centering
  {\fontsize{30}{36}\selectfont\bfseries\color{themeblue}企业短信知识题库(学习测评版)}\par
  \vspace{1.1em}
  {\Large 全知识点覆盖 · 可用于周测/岗前测/复盘测}\par
  \vspace{1.5em}
  \begin{keybox}
  \mystrong{设计原则(基于记忆科学)}\par
  \begin{itemize}
    \item 测试效应:通过做题强化记忆提取路径。
    \item 间隔重复:D0、D1、D3、D7、D14、D30滚动复习。
    \item 交错练习:概念题、规则题、场景题、计算题混合训练。
    \item 生成效应:先独立作答,再看答案解析。
    \item 难度递进:从识记到应用到方案设计。
  \end{itemize}
  \end{keybox}
  \vfill
  \coverline{版本:V3.0(双栏题答版)}
  \coverline{规则基线:SMS-CN-RULE-v2026.02 / SMS-INTL-RULE-v2026.02 / SMS-OPS-RULE-v2026.02}
  \coverline{匿名等级:Release-L2(题目全部使用匿名案例)}
  \coverline{整理时间:2026年2月8日}
\end{titlepage}

\tableofcontents
\clearpage

\chapter{使用说明与复习节奏}

\section{训练方式}
\begin{enumerate}
  \item 第1轮(D0):先做第2--5章,不看答案。
  \item 第2轮(D1):只重做错题与不确定题。
  \item 第3轮(D3):做第6章闪卡快问快答。
  \item 第4轮(D7):重做第4--5章场景与计算题。
  \item 第5轮(D14):全卷抽测(至少60题)。
  \item 第6轮(D30):闭卷复盘,目标正确率$\geq90\%$。
\end{enumerate}

\section{评分建议}
\begin{itemize}
  \item 单选题:每题1分。
  \item 多选题:每题2分(全对得分,漏选错选不得分)。
  \item 判断改错:每题1分(判断0.5+改错0.5)。
  \item 场景与计算:每题4分。
\end{itemize}

\chapter{出版级作答口径(本版新增)}

\section{术语统一(阅卷标准)}
\begin{itemize}
  \item 统一写“子端口(SubID)”,不以“扩展码/后缀码”作标准答案主写法。
  \item 统一写“携号转网”,不以“企业号转码”作标准答案主写法。
  \item 统一写“会员营销短信”,不以“会销”作标准答案主写法。
  \item 统一写“状态回执(成功/失败)”,“未知”仅表示暂未返回。
\end{itemize}

\section{规则版本(判分基线)}
\begin{longtable}{P{0.3\textwidth}P{0.24\textwidth}P{0.32\textwidth}}
\toprule
\mystrong{规则包} & \mystrong{版本} & \mystrong{判分覆盖} \\
\midrule
中国短信规则基线 & SMS-CN-RULE-v2026.02 & 计费、签名、回执、营销限制、风控 \\
国际短信规则基线 & SMS-INTL-RULE-v2026.02 & Sender ID、回填率、国家差异 \\
运营交付规则基线 & SMS-OPS-RULE-v2026.02 & 接入、压测、回执策略、重保 \\
\bottomrule
\end{longtable}

\section{规则适用声明}
本题库中的标准答案与判分口径,统一适用版本基线:
SMS-CN-RULE-v2026.02、SMS-INTL-RULE-v2026.02、SMS-OPS-RULE-v2026.02。

\section{匿名策略(题面合规)}
\begin{itemize}
  \item 题面统一使用“客户A/B/C”等匿名标识,不出现实名客户。
  \item 场景描述仅保留行业特征与业务特征,避免可逆识别信息。
  \item 解析中不出现真实客户名、私有项目代号和内部群信息。
\end{itemize}

\chapter{A卷:单项选择题(题答对照,左题右答)}
\begin{longtable}{P{0.64\textwidth}P{0.3\textwidth}}
\toprule
\mystrong{题目(含选项)} & \mystrong{答案与解析} \\
\midrule
\textbf{1.} 企业短信本质上属于哪类关系?\par \textbf{A.} C2C\par \textbf{B.} B2C\par \textbf{C.} B2B\par \textbf{D.} G2C & \ansline{B}\par \expline{依据课程规则,正确项是“B2C”。} \\
\midrule
\textbf{2.} 国内短信签名的标准格式是:\par \textbf{A.} (签名)\par \textbf{B.} [签名]\par \textbf{C.} \texttt{【签名】}\par \textbf{D.} \texttt{<签名>} & \ansline{C}\par \expline{依据课程规则,正确项是“\texttt{【签名】}”。} \\
\midrule
\textbf{3.} 下列哪项不是可用于签名报备的合规主体(课程口径)?\par \textbf{A.} 企业全称\par \textbf{B.} 合规简称\par \textbf{C.} 申请中的商标\par \textbf{D.} 已核准商标 & \ansline{C}\par \expline{依据课程规则,正确项是“申请中的商标”。} \\
\midrule
\textbf{4.} 一个子端口与签名的关系是:\par \textbf{A.} 多对多\par \textbf{B.} 一对一\par \textbf{C.} 一对多\par \textbf{D.} 多对一 & \ansline{B}\par \expline{依据课程规则,正确项是“一对一”。} \\
\midrule
\textbf{5.} 营销短信统一退订尾缀是:\par \textbf{A.} 退订回T\par \textbf{B.} 拒收请回复R\par \textbf{C.} 回复0退订\par \textbf{D.} 回复TD & \ansline{B}\par \expline{依据课程规则,正确项是“拒收请回复R”。} \\
\midrule
\textbf{6.} 短信长度140字的计费条数应为:\par \textbf{A.} 1条\par \textbf{B.} 2条\par \textbf{C.} 3条\par \textbf{D.} 4条 & \ansline{C}\par \expline{依据课程规则,正确项是“3条”。} \\
\midrule
\textbf{7.} 计费字符中,以下哪项说法正确?\par \textbf{A.} 签名不计费\par \textbf{B.} 空格不计费\par \textbf{C.} 标点不计费\par \textbf{D.} 以上都不对 & \ansline{D}\par \expline{依据课程规则,正确项是“以上都不对”。} \\
\midrule
\textbf{8.} 关于“未知状态”,正确的是:\par \textbf{A.} 最终状态之一\par \textbf{B.} 无意义状态\par \textbf{C.} 暂未返回最终状态\par \textbf{D.} 一定失败 & \ansline{C}\par \expline{依据课程规则,正确项是“暂未返回最终状态”。} \\
\midrule
\textbf{9.} 课程口径中,对账状态通常以多久后为准?\par \textbf{A.} 12小时\par \textbf{B.} 24小时\par \textbf{C.} 48小时\par \textbf{D.} 72小时 & \ansline{D}\par \expline{依据课程规则,正确项是“72小时”。} \\
\midrule
\textbf{10.} 下列哪项最强调秒级时效?\par \textbf{A.} 会员营销\par \textbf{B.} 验证码\par \textbf{C.} 节日祝福\par \textbf{D.} 品牌宣传 & \ansline{B}\par \expline{依据课程规则,正确项是“验证码”。} \\
\midrule
\textbf{11.} 下列哪类客户通常对“固定尾号+总长度”更敏感?\par \textbf{A.} 小微商户\par \textbf{B.} 个人开发者\par \textbf{C.} 大型政企/国央企\par \textbf{D.} 校园社团 & \ansline{C}\par \expline{依据课程规则,正确项是“大型政企/国央企”。} \\
\midrule
\textbf{12.} 三网合一中的“三网”是指:\par \textbf{A.} 电商三平台\par \textbf{B.} 三个数据中心\par \textbf{C.} 移动联通电信\par \textbf{D.} 三个省份 & \ansline{C}\par \expline{依据课程规则,正确项是“移动联通电信”。} \\
\midrule
\textbf{13.} 码号证获取后要先做什么才可用于实际发送?\par \textbf{A.} 充值\par \textbf{B.} 落地\par \textbf{C.} 拉群\par \textbf{D.} 投诉备案 & \ansline{B}\par \expline{依据课程规则,正确项是“落地”。} \\
\midrule
\textbf{14.} 失败返还最典型对应哪类结算模式?\par \textbf{A.} 预付费\par \textbf{B.} 后付费\par \textbf{C.} 分期\par \textbf{D.} 年付 & \ansline{A}\par \expline{依据课程规则,正确项是“预付费”。} \\
\midrule
\textbf{15.} 用户回复R后平台通常会执行:\par \textbf{A.} 二次营销\par \textbf{B.} 加入退订黑名单\par \textbf{C.} 自动拉白\par \textbf{D.} 忽略上行 & \ansline{B}\par \expline{依据课程规则,正确项是“加入退订黑名单”。} \\
\midrule
\textbf{16.} 以下哪项更可能导致“成功率低但非平台故障”?\par \textbf{A.} 大量空号停机号\par \textbf{B.} 代码异常\par \textbf{C.} 通道断连\par \textbf{D.} 机房断电 & \ansline{A}\par \expline{依据课程规则,正确项是“大量空号停机号”。} \\
\midrule
\textbf{17.} 客户只要求测试平台接入能力时,常见压测方式是:\par \textbf{A.} 真机拨测\par \textbf{B.} 通道配空\par \textbf{C.} 全量上生产\par \textbf{D.} 仅人工审核 & \ansline{B}\par \expline{依据课程规则,正确项是“通道配空”。} \\
\midrule
\textbf{18.} 全链路压测常用的号码策略是:\par \textbf{A.} 全真号\par \textbf{B.} 全白名单\par \textbf{C.} 空号压测\par \textbf{D.} 内部号 & \ansline{C}\par \expline{依据课程规则,正确项是“空号压测”。} \\
\midrule
\textbf{19.} 以下哪项最可能需要“限流回推状态”?\par \textbf{A.} 小客户日发几十条\par \textbf{B.} 大客户峰值QPS很高\par \textbf{C.} 新注册客户\par \textbf{D.} 静态通知 & \ansline{B}\par \expline{依据课程规则,正确项是“大客户峰值QPS很高”。} \\
\midrule
\textbf{20.} 国际短信品牌识别核心字段是:\par \textbf{A.} Sender ID\par \textbf{B.} Signature ID\par \textbf{C.} Route ID\par \textbf{D.} Channel ID & \ansline{A}\par \expline{依据课程规则,正确项是“Sender ID”。} \\
\midrule
\textbf{21.} 国际验证码最常见核心效果指标是:\par \textbf{A.} UV\par \textbf{B.} PV\par \textbf{C.} 回填率\par \textbf{D.} 打开率 & \ansline{C}\par \expline{依据课程规则,正确项是“回填率”。} \\
\midrule
\textbf{22.} 下列哪项是平台侧常见风控策略?\par \textbf{A.} 黑名单\par \textbf{B.} 关键词\par \textbf{C.} 单号码频控\par \textbf{D.} 以上都是 & \ansline{D}\par \expline{依据课程规则,正确项是“以上都是”。} \\
\midrule
\textbf{23.} 会员营销短信的前提是:\par \textbf{A.} 任何手机号都可\par \textbf{B.} 只要买量就可\par \textbf{C.} 用户与企业存在会员关系与授权\par \textbf{D.} 只要是促销季 & \ansline{C}\par \expline{依据课程规则,正确项是“用户与企业存在会员关系与授权”。} \\
\midrule
\textbf{24.} 下列哪种情况最可能触发“多签名”风险?\par \textbf{A.} 正文含数字\par \textbf{B.} 正文再使用方头括号\par \textbf{C.} 正文有空格\par \textbf{D.} 正文有英文 & \ansline{B}\par \expline{依据课程规则,正确项是“正文再使用方头括号”。} \\
\midrule
\textbf{25.} 平台中“提交回执”指:\par \textbf{A.} 终端已收到短信\par \textbf{B.} 运营商已计费\par \textbf{C.} 平台已收到客户提交\par \textbf{D.} 用户已回复 & \ansline{C}\par \expline{依据课程规则,正确项是“平台已收到客户提交”。} \\
\midrule
\textbf{26.} 下列哪类短信通常不宜重人工审核?\par \textbf{A.} 会员营销\par \textbf{B.} 高危金融营销\par \textbf{C.} 验证码\par \textbf{D.} 节日活动 & \ansline{C}\par \expline{依据课程规则,正确项是“验证码”。} \\
\midrule
\textbf{27.} 以下哪个不是典型投诉入口?\par \textbf{A.} 12321\par \textbf{B.} 运营商客服\par \textbf{C.} 通管局\par \textbf{D.} 气象台 & \ansline{D}\par \expline{依据课程规则,正确项是“气象台”。} \\
\midrule
\textbf{28.} 电商客户在618、双11时更关注:\par \textbf{A.} 静态美工\par \textbf{B.} QPS承载与稳定性\par \textbf{C.} 语音资费\par \textbf{D.} 国际区号 & \ansline{B}\par \expline{依据课程规则,正确项是“QPS承载与稳定性”。} \\
\midrule
\textbf{29.} 课程中“有效号码”概念强调的是:\par \textbf{A.} 任何格式正确号码\par \textbf{B.} 可真实触达并可接收短信的号码\par \textbf{C.} 白名单号码\par \textbf{D.} 短号 & \ansline{B}\par \expline{依据课程规则,正确项是“可真实触达并可接收短信的号码”。} \\
\midrule
\textbf{30.} 携号转网的含义是:\par \textbf{A.} 改手机号\par \textbf{B.} 改签名\par \textbf{C.} 号码不变、运营商归属变更\par \textbf{D.} 改套餐 & \ansline{C}\par \expline{依据课程规则,正确项是“号码不变、运营商归属变更”。} \\
\midrule
\textbf{31.} 有携转库时,平台的更优做法是:\par \textbf{A.} 永远按号段发\par \textbf{B.} 按当前归属网发\par \textbf{C.} 随机发\par \textbf{D.} 全部失败 & \ansline{B}\par \expline{依据课程规则,正确项是“按当前归属网发”。} \\
\midrule
\textbf{32.} 影响利润最直接的四因子中不包括:\par \textbf{A.} 单价\par \textbf{B.} 计费口径\par \textbf{C.} 通道复杂度\par \textbf{D.} 办公区楼层 & \ansline{D}\par \expline{依据课程规则,正确项是“办公区楼层”。} \\
\midrule
\textbf{33.} 对小微客户更推荐的接入方式通常是:\par \textbf{A.} 深度定制平台\par \textbf{B.} Web自服务\par \textbf{C.} 私有化全套\par \textbf{D.} 仅线下导入 & \ansline{B}\par \expline{依据课程规则,正确项是“Web自服务”。} \\
\midrule
\textbf{34.} 以下哪项最可能导致通道健康受损?\par \textbf{A.} 投诉超限\par \textbf{B.} 日常优化\par \textbf{C.} 账号加白\par \textbf{D.} 成功率高 & \ansline{A}\par \expline{依据课程规则,正确项是“投诉超限”。} \\
\midrule
\textbf{35.} 大客户为何常需要状态回执“限速回推”?\par \textbf{A.} 省流量\par \textbf{B.} 回执处理系统承载有限\par \textbf{C.} 便于营销\par \textbf{D.} 无意义 & \ansline{B}\par \expline{依据课程规则,正确项是“回执处理系统承载有限”。} \\
\midrule
\textbf{36.} 下列哪项是阅信的典型优势?\par \textbf{A.} 纯文本无交互\par \textbf{B.} 卡片化展示与跳转能力\par \textbf{C.} 不需要报备链接\par \textbf{D.} 仅支持苹果 & \ansline{B}\par \expline{依据课程规则,正确项是“卡片化展示与跳转能力”。} \\
\midrule
\textbf{37.} 阅信在iOS上的常见体验是:\par \textbf{A.} 自动卡片直开\par \textbf{B.} 常需点击链接后呈现\par \textbf{C.} 彻底无法接收\par \textbf{D.} 自动转语音 & \ansline{B}\par \expline{依据课程规则,正确项是“常需点击链接后呈现”。} \\
\midrule
\textbf{38.} 富媒体短信相较文本短信最典型特点是:\par \textbf{A.} 更便宜\par \textbf{B.} 展示更丰富但通常更贵\par \textbf{C.} 不支持图文\par \textbf{D.} 仅通知可用 & \ansline{B}\par \expline{依据课程规则,正确项是“展示更丰富但通常更贵”。} \\
\midrule
\textbf{39.} 下列关于“未知率”说法正确的是:\par \textbf{A.} 越高越好\par \textbf{B.} 正常应较低且随时间收敛\par \textbf{C.} 永不变化\par \textbf{D.} 与链路无关 & \ansline{B}\par \expline{依据课程规则,正确项是“正常应较低且随时间收敛”。} \\
\midrule
\textbf{40.} 客户要求“主动拉取状态”,平台通常会重点评估:\par \textbf{A.} 客户字体偏好\par \textbf{B.} 资源占用与安全隔离\par \textbf{C.} 客户Logo颜色\par \textbf{D.} 话术风格 & \ansline{B}\par \expline{依据课程规则,正确项是“资源占用与安全隔离”。} \\
\midrule
\textbf{41.} 下列哪项最符合“批量测试”定义?\par \textbf{A.} 只发1条验证码\par \textbf{B.} 切一部分真实业务观察多天\par \textbf{C.} 不做任何测试\par \textbf{D.} 只看报价 & \ansline{B}\par \expline{依据课程规则,正确项是“切一部分真实业务观察多天”。} \\
\midrule
\textbf{42.} 渠道客户合作的核心通常是:\par \textbf{A.} 装修风格\par \textbf{B.} 资源能力与成本效率\par \textbf{C.} 节日礼物\par \textbf{D.} 办公地点 & \ansline{B}\par \expline{渠道合作本质是“资源与成本效率匹配”,而不是品牌或行政因素。} \\
\midrule
\textbf{43.} 下列哪项最能体现“平台级交付能力”?\par \textbf{A.} 临时群聊\par \textbf{B.} 私有化部署与持续运维\par \textbf{C.} 单次报价\par \textbf{D.} 单次演示 & \ansline{B}\par \expline{依据课程规则,正确项是“私有化部署与持续运维”。} \\
\midrule
\textbf{44.} 对客户承诺成功率时最正确表述是:\par \textbf{A.} 永远100\%\par \textbf{B.} 不看号码质量\par \textbf{C.} 在有效号码前提下承诺\par \textbf{D.} 不做任何说明 & \ansline{C}\par \expline{成功率承诺必须以“有效号码、可触达号码”作为前提条件。} \\
\midrule
\textbf{45.} 以下哪项属于“引流信息”需报备要素?\par \textbf{A.} 链接与电话号码\par \textbf{B.} 仅标点\par \textbf{C.} 仅签名\par \textbf{D.} 仅空格 & \ansline{A}\par \expline{引流信息的核心是“可引导触达”的要素,典型就是链接与电话号码。} \\
\midrule
\textbf{46.} 若客户每天发送量极低,最合理服务策略是:\par \textbf{A.} 强制私有化\par \textbf{B.} 标准化自服务+预付优先\par \textbf{C.} 先压测1万QPS\par \textbf{D.} 关闭回执 & \ansline{B}\par \expline{依据课程规则,正确项是“标准化自服务+预付优先”。} \\
\midrule
\textbf{47.} 错误码释义表最准确的说法是:\par \textbf{A.} 一定100\%唯一准确\par \textbf{B.} 仅作参考,需结合通道核实\par \textbf{C.} 完全没用\par \textbf{D.} 与运营无关 & \ansline{B}\par \expline{依据课程规则,正确项是“仅作参考,需结合通道核实”。} \\
\midrule
\textbf{48.} 对于高危营销账号,单号码频控策略通常是:\par \textbf{A.} 更宽松\par \textbf{B.} 更严格\par \textbf{C.} 与验证码一样\par \textbf{D.} 不设限制 & \ansline{B}\par \expline{依据课程规则,正确项是“更严格”。} \\
\midrule
\textbf{49.} 下列哪项最符合“测试效应”学习法?\par \textbf{A.} 只看不做题\par \textbf{B.} 做题后再看解析\par \textbf{C.} 永远不复习\par \textbf{D.} 只收藏 & \ansline{B}\par \expline{测试效应强调“先提取再反馈”,即先作答、再核对解析。} \\
\midrule
\textbf{50.} 课程建议的复习节奏中不包括:\par \textbf{A.} D1复习\par \textbf{B.} D3复习\par \textbf{C.} D7复习\par \textbf{D.} D365单次复习 & \ansline{D}\par \expline{本课节奏为 D0/D1/D3/D7/D14/D30,不包含 D365 单次复习。} \\
\midrule
\textbf{51.} 若客户投诉“我不是会员却收到营销”,第一风险归因是:\par \textbf{A.} 计费过高\par \textbf{B.} 隐私与合规风险\par \textbf{C.} 接口版本\par \textbf{D.} 字体问题 & \ansline{B}\par \expline{依据课程规则,正确项是“隐私与合规风险”。} \\
\midrule
\textbf{52.} 下列哪项最体现“销售前置价值”?\par \textbf{A.} 只谈价格\par \textbf{B.} 提前问清码号、量级、投诉、回执、QPS\par \textbf{C.} 只发合同\par \textbf{D.} 只拉技术群 & \ansline{B}\par \expline{前置把关键变量问清,才能让报价、资源和上线方案一次性做对。} \\
\midrule
\textbf{53.} 平台对验证码轰炸的核心防护是:\par \textbf{A.} 提高价格\par \textbf{B.} 防轰炸频控策略\par \textbf{C.} 取消回执\par \textbf{D.} 关闭上行 & \ansline{B}\par \expline{依据课程规则,正确项是“防轰炸频控策略”。} \\
\midrule
\textbf{54.} 国际短信中可能存在的额外成本是:\par \textbf{A.} 国家报备注册费/月租\par \textbf{B.} 机房水费\par \textbf{C.} 办公室停车费\par \textbf{D.} 内网设备折旧 & \ansline{A}\par \expline{国际路由常见附加成本是国家侧注册费、品牌报备费或月租费。} \\
\midrule
\textbf{55.} 下列哪个更像“运营持续调优”工作?\par \textbf{A.} 一次性开账号后不管\par \textbf{B.} 根据投诉和成功率动态调黑白名单与通道权重\par \textbf{C.} 仅看月报\par \textbf{D.} 仅看合同 & \ansline{B}\par \expline{依据课程规则,正确项是“根据投诉和成功率动态调黑白名单与通道权重”。} \\
\midrule
\textbf{56.} 客户要求“状态只拉不推”时,不应忽略的风险是:\par \textbf{A.} 客户忘记拉取导致堆积\par \textbf{B.} 文案变好\par \textbf{C.} 推送更快\par \textbf{D.} 无风险 & \ansline{A}\par \expline{若客户拉取任务异常或漏拉,状态会在平台堆积并影响后续查询与核对。} \\
\midrule
\textbf{57.} 下列哪项属于“上线前必须确认项”?\par \textbf{A.} 头像尺寸\par \textbf{B.} 签名和引流报备结果\par \textbf{C.} 名片样式\par \textbf{D.} 工位数量 & \ansline{B}\par \expline{依据课程规则,正确项是“签名和引流报备结果”。} \\
\midrule
\textbf{58.} 下列关于私有化部署客户的特点,正确的是:\par \textbf{A.} 粘性通常更低\par \textbf{B.} 粘性通常更高\par \textbf{C.} 不需要运维\par \textbf{D.} 只做一次性交付 & \ansline{B}\par \expline{依据课程规则,正确项是“粘性通常更高”。} \\
\midrule
\textbf{59.} 最能体现“交错练习”的做法是:\par \textbf{A.} 连续做100道同类型记忆题\par \textbf{B.} 概念题与计算题、场景题混做\par \textbf{C.} 只看答案\par \textbf{D.} 只听课 & \ansline{B}\par \expline{依据课程规则,正确项是“概念题与计算题、场景题混做”。} \\
\midrule
\textbf{60.} 对外发布前,关于版本一致性的正确做法是:\par \textbf{A.} 只改封面不改元信息\par \textbf{B.} PDF元信息与封面版本保持一致\par \textbf{C.} 版本号可省略\par \textbf{D.} 仅对内文标注版本 & \ansline{B}\par \expline{依据课程规则,正确项是“PDF元信息与封面版本保持一致”。} \\
\bottomrule
\end{longtable}
\chapter{B卷:多项选择题(题答对照,左题右答)}
\begin{longtable}{P{0.64\textwidth}P{0.3\textwidth}}
\toprule
\mystrong{题目(含选项)} & \mystrong{答案与解析} \\
\midrule
\textbf{1.} 国内企业短信签名报备可用来源通常包括( )。\par \textbf{A.} 企业全称\par \textbf{B.} 合规简称\par \textbf{C.} 已核准商标\par \textbf{D.} 申请中商标 & \ansline{ABC}\par \expline{本题应选择 A、B、C,对应题干要求的完整要点集合。} \\
\midrule
\textbf{2.} 影响成功率的常见因素有( )。\par \textbf{A.} 空号停机\par \textbf{B.} 黑名单命中\par \textbf{C.} 关键词拦截\par \textbf{D.} 终端无信号 & \ansline{ABCD}\par \expline{本题应选择 A、B、C、D,对应题干要求的完整要点集合。} \\
\midrule
\textbf{3.} 营销短信合规关键点包括( )。\par \textbf{A.} 会员前提\par \textbf{B.} 退订口径\par \textbf{C.} 时间窗控制\par \textbf{D.} 频控策略 & \ansline{ABCD}\par \expline{本题应选择 A、B、C、D,对应题干要求的完整要点集合。} \\
\midrule
\textbf{4.} 客户接入前销售应重点确认( )。\par \textbf{A.} 业务场景和量级\par \textbf{B.} 码号需求\par \textbf{C.} 投诉历史\par \textbf{D.} 回执方式 & \ansline{ABCD}\par \expline{本题应选择 A、B、C、D,对应题干要求的完整要点集合。} \\
\midrule
\textbf{5.} 压测前需确认( )。\par \textbf{A.} 目标QPS\par \textbf{B.} 测试时段与时长\par \textbf{C.} 压测模式\par \textbf{D.} 是否影响线上业务 & \ansline{ABCD}\par \expline{本题应选择 A、B、C、D,对应题干要求的完整要点集合。} \\
\midrule
\textbf{6.} 状态回执策略可包括( )。\par \textbf{A.} 实时推送\par \textbf{B.} 限速推送\par \textbf{C.} 客户主动拉取\par \textbf{D.} 关闭所有回执 & \ansline{ABC}\par \expline{本题应选择 A、B、C,对应题干要求的完整要点集合。} \\
\midrule
\textbf{7.} 以下哪些属于平台风控机制( )。\par \textbf{A.} 黑名单\par \textbf{B.} 白名单\par \textbf{C.} 关键词\par \textbf{D.} 防轰炸 & \ansline{ABCD}\par \expline{本题应选择 A、B、C、D,对应题干要求的完整要点集合。} \\
\midrule
\textbf{8.} 下列哪些属于“引流信息”需报备项( )。\par \textbf{A.} 链接\par \textbf{B.} 电话号码\par \textbf{C.} 纯标点\par \textbf{D.} 无内容空格 & \ansline{AB}\par \expline{本题应选择 A、B,对应题干要求的完整要点集合。} \\
\midrule
\textbf{9.} 国际短信中常见的国家差异项有( )。\par \textbf{A.} Sender ID规则\par \textbf{B.} 退订规则\par \textbf{C.} 报备材料\par \textbf{D.} 费用结构 & \ansline{ABCD}\par \expline{本题应选择 A、B、C、D,对应题干要求的完整要点集合。} \\
\midrule
\textbf{10.} 下列哪些场景更强调通知而非营销( )。\par \textbf{A.} 动账提醒\par \textbf{B.} 物流取件码\par \textbf{C.} 系统维护通知\par \textbf{D.} 双11促销 & \ansline{ABC}\par \expline{本题应选择 A、B、C,对应题干要求的完整要点集合。} \\
\midrule
\textbf{11.} 长短信对账争议常与哪些因素相关( )。\par \textbf{A.} 分片计费\par \textbf{B.} 补发策略\par \textbf{C.} 回执口径\par \textbf{D.} 容差规则 & \ansline{ABCD}\par \expline{本题应选择 A、B、C、D,对应题干要求的完整要点集合。} \\
\midrule
\textbf{12.} 可用于说明“未知不是最终状态”的证据有( )。\par \textbf{A.} 72小时内未知会收敛\par \textbf{B.} 未知可转成功/失败\par \textbf{C.} 未知永不变化\par \textbf{D.} 未知本质是暂未返回 & \ansline{ABD}\par \expline{本题应选择 A、B、D,对应题干要求的完整要点集合。} \\
\midrule
\textbf{13.} 对大中直客的服务重点通常包括( )。\par \textbf{A.} 重保\par \textbf{B.} 快速响应\par \textbf{C.} 定制能力\par \textbf{D.} 数据报告 & \ansline{ABCD}\par \expline{本题应选择 A、B、C、D,对应题干要求的完整要点集合。} \\
\midrule
\textbf{14.} 以下哪些可作为小微客户策略( )。\par \textbf{A.} Web自服务\par \textbf{B.} 预付优先\par \textbf{C.} 标准流程\par \textbf{D.} 全部私有化 & \ansline{ABC}\par \expline{本题应选择 A、B、C,对应题干要求的完整要点集合。} \\
\midrule
\textbf{15.} 下列哪些属于投诉治理动作( )。\par \textbf{A.} 收集会员证明\par \textbf{B.} 核实隐私授权\par \textbf{C.} 优化频控与黑名单策略\par \textbf{D.} 长期忽略投诉 & \ansline{ABC}\par \expline{本题应选择 A、B、C,对应题干要求的完整要点集合。} \\
\midrule
\textbf{16.} 可导致“映射释义不完全准确”的原因有( )。\par \textbf{A.} 三方通道同码异义\par \textbf{B.} 运营商同码多义\par \textbf{C.} 通道策略差异\par \textbf{D.} 客户接口差异 & \ansline{ABCD}\par \expline{本题应选择 A、B、C、D,对应题干要求的完整要点集合。} \\
\midrule
\textbf{17.} 以下哪些属于“上线前必须完成”的内容( )。\par \textbf{A.} 报备完成\par \textbf{B.} 回执策略确认\par \textbf{C.} 风控参数确认\par \textbf{D.} 应急联系人确认 & \ansline{ABCD}\par \expline{本题应选择 A、B、C、D,对应题干要求的完整要点集合。} \\
\midrule
\textbf{18.} 阅信相较纯文本可新增的能力有( )。\par \textbf{A.} 卡片化展示\par \textbf{B.} 一键跳APP\par \textbf{C.} 点击追踪\par \textbf{D.} 解析统计 & \ansline{ABCD}\par \expline{本题应选择 A、B、C、D,对应题干要求的完整要点集合。} \\
\midrule
\textbf{19.} 关于携号转网,正确的有( )。\par \textbf{A.} 号段与当前归属网可能不一致\par \textbf{B.} 有携转库时可按当前归属网投递\par \textbf{C.} MO回传在部分链路有差异\par \textbf{D.} 与三网合一无关 & \ansline{ABC}\par \expline{本题应选择 A、B、C,对应题干要求的完整要点集合。} \\
\midrule
\textbf{20.} 平台后台运营的主要工作包括( )。\par \textbf{A.} 通道池调度\par \textbf{B.} 监控告警\par \textbf{C.} 投诉控制\par \textbf{D.} 平台迭代 & \ansline{ABCD}\par \expline{本题应选择 A、B、C、D,对应题干要求的完整要点集合。} \\
\midrule
\textbf{21.} 计费相关客户高频问题通常有( )。\par \textbf{A.} 签名是否计费\par \textbf{B.} 括号是否计费\par \textbf{C.} 140字为何3条\par \textbf{D.} 空格是否计费 & \ansline{ABCD}\par \expline{本题应选择 A、B、C、D,对应题干要求的完整要点集合。} \\
\midrule
\textbf{22.} 对“回填率”理解正确的有( )。\par \textbf{A.} 多用于国际验证码场景\par \textbf{B.} 是实际填写验证码比例\par \textbf{C.} 等同于平台提交成功率\par \textbf{D.} 可用于评估链路质量 & \ansline{ABD}\par \expline{本题应选择 A、B、D,对应题干要求的完整要点集合。} \\
\midrule
\textbf{23.} 影响利润的关键可控动作包括( )。\par \textbf{A.} 优化计费口径\par \textbf{B.} 提升有效触达\par \textbf{C.} 合理匹配通道成本\par \textbf{D.} 强化客户结构管理 & \ansline{ABCD}\par \expline{本题应选择 A、B、C、D,对应题干要求的完整要点集合。} \\
\midrule
\textbf{24.} 属于脑科学高效学习策略的有( )。\par \textbf{A.} 间隔重复\par \textbf{B.} 主动回忆\par \textbf{C.} 交错练习\par \textbf{D.} 只被动阅读 & \ansline{ABC}\par \expline{本题应选择 A、B、C,对应题干要求的完整要点集合。} \\
\midrule
\textbf{25.} 下列哪些情况应立即升级协同(销售+运营+技术)( )。\par \textbf{A.} 大客户压测上万QPS\par \textbf{B.} 大面积成功率异常\par \textbf{C.} 投诉突增\par \textbf{D.} 关键客户节前重保 & \ansline{ABCD}\par \expline{本题应选择 A、B、C、D,对应题干要求的完整要点集合。} \\
\bottomrule
\end{longtable}
\chapter{C卷:判断改错题(题答对照,左题右答)}
\begin{longtable}{P{0.6\textwidth}P{0.34\textwidth}}
\toprule
\mystrong{题目} & \mystrong{答案与改错} \\
\midrule
\textbf{1.} “未知状态就是第三种最终状态。”(对/错,并改错) & 错。未知是“暂未返回”,非终态。 \\
\midrule
\textbf{2.} “营销短信可以不给退订口径。”(对/错,并改错) & 错。营销短信必须有统一退订口径。 \\
\midrule
\textbf{3.} “140字短信按2条计费。”(对/错,并改错) & 错。140字按67分片,计3条。 \\
\midrule
\textbf{4.} “一个子端口可同时对应多个签名。”(对/错,并改错) & 错。子端口与签名是一对一。 \\
\midrule
\textbf{5.} “正文再次使用方头括号不会有风险。”(对/错,并改错) & 错。可能触发多签名风险。 \\
\midrule
\textbf{6.} “只要有码号证就能直接发短信。”(对/错,并改错) & 错。需完成运营商落地后才能发送。 \\
\midrule
\textbf{7.} “三网合一一定比普通资源便宜。”(对/错,并改错) & 错。三网合一通常更贵。 \\
\midrule
\textbf{8.} “验证码短信时效不敏感。”(对/错,并改错) & 错。验证码对时效高度敏感。 \\
\midrule
\textbf{9.} “白名单号码也会完全受日频限制。”(对/错,并改错) & 错。白名单可放宽部分限制。 \\
\midrule
\textbf{10.} “黑名单都可以一键解除。”(对/错,并改错) & 错。黑名单分级,非全部可解。 \\
\midrule
\textbf{11.} “国际短信各国规则基本一样。”(对/错,并改错) & 错。各国规则差异显著。 \\
\midrule
\textbf{12.} “回填率主要用于国际验证码评估。”(对/错,并改错) & 对。 \\
\midrule
\textbf{13.} “客户主动拉取状态不会占用平台资源。”(对/错,并改错) & 错。主动拉取会占用平台资源。 \\
\midrule
\textbf{14.} “批量测试通常要跑一段真实业务观察。”(对/错,并改错) & 对。 \\
\midrule
\textbf{15.} “小微客户一般更适合先上私有化部署。”(对/错,并改错) & 错。小微客户一般先用自服务。 \\
\midrule
\textbf{16.} “成功率承诺可以不考虑号码质量。”(对/错,并改错) & 错。需以有效号码为前提。 \\
\midrule
\textbf{17.} “引流链接不需要报备。”(对/错,并改错) & 错。引流链接需报备。 \\
\midrule
\textbf{18.} “高危营销的频控通常会更严格。”(对/错,并改错) & 对。 \\
\midrule
\textbf{19.} “投诉治理与销售无关,只是运营的事。”(对/错,并改错) & 错。销售需协助投诉证据链。 \\
\midrule
\textbf{20.} “测试效应强调做题本身能强化记忆。”(对/错,并改错) & 对。 \\
\bottomrule
\end{longtable}
\chapter{D卷:场景题与计算题(题答对照,左题右答)}
\section{场景题(8题)}
\begin{longtable}{P{0.6\textwidth}P{0.34\textwidth}}
\toprule
\mystrong{题目} & \mystrong{参考答案} \\
\midrule
\textbf{1.} 客户A说“我们不需要任何码号要求”,上线后又要求“固定尾号+总长不超11位+三网一致”。你作为销售如何补救并与运营协同? & 先补充需求澄清单并与客户确认;再由运营评估可用码号池和三网一致性成本,形成变更报价与交期。 \\
\midrule
\textbf{2.} 客户B为高频营销行业,投诉持续升高,成功率也在下降。请给出“合规+成功率+成本”三目标下的调优方案。 & 先控投诉(会员与模板审计、频控收紧、黑名单策略);再提升成功率(通道权重与地区策略调优);最后回看成本并做分层路由。 \\
\midrule
\textbf{3.} 客户C要求“只拉状态不推状态”,并在一周后反馈“状态数据不全”。请分析最可能原因与修复方案。 & 排查是否“未拉取、拉取失败、拉取窗口不一致”;补充拉取监控告警、失败重试与数据留存策略。 \\
\midrule
\textbf{4.} 客户D做618大促,计划2小时内持续3000 QPS。请给出接入前检查项与压测方案。 & 明确QPS、时段、时长、压测模式、回执模式;先压测再灰度扩量,并设置应急回滚与专人值守。 \\
\midrule
\textbf{5.} 客户E反馈“同一批数据,上午查和下午查成功率不一样”。请用状态机制解释。 & 解释未知状态会在72小时内收敛,上午与下午查询窗口不同导致结果波动。 \\
\midrule
\textbf{6.} 客户F做国际验证码,提出“为什么成功率还行但回填率低”。给出至少4个排查维度。 & 排查通道质量、时延、国家规则、终端可达、验证码有效期与页面体验。 \\
\midrule
\textbf{7.} 客户G坚持营销短信晚11点发。给出两种平台处理策略,并分析业务利弊。 & 两种策略:直接失败或延时到次日窗口;前者合规最稳,后者业务体验更好但需客户接受延迟。 \\
\midrule
\textbf{8.} 客户H提出“同一个签名要绑定多个活动链接”。你如何设计子端口与引流报备方案? & 同签名多活动需要多子端口拆分;每个子端口绑定固定引流信息并完成报备。 \\
\bottomrule
\end{longtable}
\section{计算题(7题)}
\begin{longtable}{P{0.6\textwidth}P{0.34\textwidth}}
\toprule
\mystrong{题目} & \mystrong{标准答案} \\
\midrule
\textbf{1.} 某短信126字,按课程计费规则应计费多少条? & 2条($\lceil126/67\rceil=2$)。 \\
\midrule
\textbf{2.} 某客户提交10000条,72小时后成功9200、失败700、未知100(仍未回)。在“成功计费”与“失败不计费(成功+未知计费)”两种模式下分别计费多少条? & 成功计费=9200;失败不计费(成功+未知计费)=9300。 \\
\midrule
\textbf{3.} 某账号单号日上限10条。某号码当日已收8条通知,再发5条验证码,最多还能成功几条(不考虑其他限制)? & 最多2条。 \\
\midrule
\textbf{4.} 某国际验证码通道提交5000条,回填3200条,回填率是多少? & 64\%($3200/5000$)。 \\
\midrule
\textbf{5.} 某客户发140字长短信1000次,全部一次成功。按课程规则总计费条数是多少? & 3000条(每条140字计3条)。 \\
\midrule
\textbf{6.} 某客户发140字短信1000次,其中每次第一轮“1片成功1片失败”,第二轮仅补发失败片且全部成功。总计费条数是多少(按分片成功计费)? & 2000条(每次2条,1000次)。 \\
\midrule
\textbf{7.} 某运营周报显示:周一未知率2.5\%,周二0.9\%,周三0.8\%。从健康度看哪一天风险最高? & 周一风险最高。 \\
\bottomrule
\end{longtable}
\chapter{E卷:闪卡快问快答(81题,题答对照)}
\begin{longtable}{P{0.56\textwidth}P{0.38\textwidth}}
\toprule
\mystrong{问题} & \mystrong{答案} \\
\midrule
\textbf{1.} 企业短信核心关系? & B2C。 \\
\midrule
\textbf{2.} 国内签名标准格式? & \texttt{【签名】}。 \\
\midrule
\textbf{3.} 营销退订统一文案? & 拒收请回复R。 \\
\midrule
\textbf{4.} 67字以内计费规则? & $\leq67$字计1条。 \\
\midrule
\textbf{5.} 超67字拆分规则? & $>67$字按67字分片。 \\
\midrule
\textbf{6.} 140字计费条数? & 3条。 \\
\midrule
\textbf{7.} 签名是否计费? & 计费。 \\
\midrule
\textbf{8.} 空格是否计费? & 计费。 \\
\midrule
\textbf{9.} 标点是否计费? & 计费。 \\
\midrule
\textbf{10.} 未知是不是最终状态? & 不是。 \\
\midrule
\textbf{11.} 对账常用状态窗口? & 72小时。 \\
\midrule
\textbf{12.} 子端口到签名关系? & 一对一。 \\
\midrule
\textbf{13.} 一个签名可否多个端口? & 可以(一签名可多子端口)。 \\
\midrule
\textbf{14.} 什么是三网合一? & 三网发件标识一致。 \\
\midrule
\textbf{15.} 三网分别是? & 移动、联通、电信。 \\
\midrule
\textbf{16.} 码号证后下一步? & 落地成通道。 \\
\midrule
\textbf{17.} MT是什么意思? & 下行短信。 \\
\midrule
\textbf{18.} MO是什么意思? & 上行短信。 \\
\midrule
\textbf{19.} 提交回执定义? & 平台已接收请求。 \\
\midrule
\textbf{20.} 状态回执定义? & 成功/失败回执。 \\
\midrule
\textbf{21.} 会员营销前提? & 会员关系+授权。 \\
\midrule
\textbf{22.} 验证码首要指标? & 秒级时效与到达。 \\
\midrule
\textbf{23.} 通知短信典型场景? & 动账、物流、工单等。 \\
\midrule
\textbf{24.} 电商高峰期关注什么? & QPS承载与稳定性。 \\
\midrule
\textbf{25.} 黑名单作用? & 拦截高风险/投诉号。 \\
\midrule
\textbf{26.} 白名单作用? & 对测试/重保号放行。 \\
\midrule
\textbf{27.} 上行R通常触发什么? & 加入退订黑名单。 \\
\midrule
\textbf{28.} 关键词机制目的? & 内容合规拦截。 \\
\midrule
\textbf{29.} 防轰炸机制目的? & 防恶意验证码轰炸。 \\
\midrule
\textbf{30.} 高危营销频控通常如何? & 更严格。 \\
\midrule
\textbf{31.} 营销发送时间常规窗口? & 常规早8晚10。 \\
\midrule
\textbf{32.} 晚间提交营销可怎么处理? & 失败或延时到次日窗口。 \\
\midrule
\textbf{33.} 引流信息包含什么? & 链接、电话号码。 \\
\midrule
\textbf{34.} 引流信息是否需报备? & 需要报备。 \\
\midrule
\textbf{35.} 国际品牌识别字段? & Sender ID。 \\
\midrule
\textbf{36.} 国际验证码核心指标? & 回填率。 \\
\midrule
\textbf{37.} 回填率定义? & 填回验证码比例。 \\
\midrule
\textbf{38.} 有效号码定义要点? & 可触达可接收。 \\
\midrule
\textbf{39.} 空号属于有效号码吗? & 不属于。 \\
\midrule
\textbf{40.} 飞行模式会影响什么? & 影响接收(失败/延迟)。 \\
\midrule
\textbf{41.} 携号转网定义? & 号不变、归属网变。 \\
\midrule
\textbf{42.} 有携转库应按什么发? & 按当前归属网下发。 \\
\midrule
\textbf{43.} 主动拉取状态风险? & 资源占用与堆积风险。 \\
\midrule
\textbf{44.} 状态限流回推适用谁? & 高QPS大客户。 \\
\midrule
\textbf{45.} 错误码映射是否绝对准确? & 不是,只作参考。 \\
\midrule
\textbf{46.} 失败返还常见于哪类结算? & 预付费。 \\
\midrule
\textbf{47.} 大客户服务四要素? & 重保、响应、定制、报告。 \\
\midrule
\textbf{48.} 小微客户优先接入方式? & Web自服务。 \\
\midrule
\textbf{49.} 压测前必问三件事? & QPS、时段时长、压测模式。 \\
\midrule
\textbf{50.} 批测的本质? & 小规模真实业务观察。 \\
\midrule
\textbf{51.} 私有化客户粘性通常如何? & 通常更高。 \\
\midrule
\textbf{52.} 投诉治理要不要销售参与? & 要参与。 \\
\midrule
\textbf{53.} 12321是什么? & 工信部投诉受理渠道。 \\
\midrule
\textbf{54.} 通道健康受什么强影响? & 投诉指标/百投比。 \\
\midrule
\textbf{55.} 三方通道错误码特点? & 同码可能异义。 \\
\midrule
\textbf{56.} 电信错误码常见难点? & 常见同码多义。 \\
\midrule
\textbf{57.} 长短信对账为什么易争议? & 分片计费与口径差异。 \\
\midrule
\textbf{58.} 只回一条状态会带来什么风险? & 造成账单偏差风险。 \\
\midrule
\textbf{59.} 阅信与文本主要差异? & 卡片化+可跳转。 \\
\midrule
\textbf{60.} 阅信在iOS常见体验? & 常需先点链接。 \\
\midrule
\textbf{61.} 富媒体核心优势? & 展示更丰富但更贵。 \\
\midrule
\textbf{62.} 5G消息主要瓶颈之一? & 终端覆盖限制。 \\
\midrule
\textbf{63.} 平台监控至少看哪三项? & 成功率、未知率、QPS/时延。 \\
\midrule
\textbf{64.} 上线前核对至少哪三项? & 报备、回执、风控参数。 \\
\midrule
\textbf{65.} 合同里最好约定什么口径? & 计费口径与回执口径。 \\
\midrule
\textbf{66.} 销售最该提前确认什么? & 量级、码号、投诉、QPS等。 \\
\midrule
\textbf{67.} 影响利润四因子? & 单价、计费口径、通道成本、成功率。 \\
\midrule
\textbf{68.} 测试效应一句话定义? & 做题提取强化记忆。 \\
\midrule
\textbf{69.} 间隔重复一句话定义? & 分时多轮重复复习。 \\
\midrule
\textbf{70.} 交错练习一句话定义? & 概念题与计算题混练。 \\
\midrule
\textbf{71.} D1复习做什么? & 重做错题和不确定题。 \\
\midrule
\textbf{72.} D7复习做什么? & 重做场景题与计算题。 \\
\midrule
\textbf{73.} D30复习目标正确率? & $\geq90\%$。 \\
\midrule
\textbf{74.} 如果未知率突然升高先查哪? & 先查通道与回执收敛。 \\
\midrule
\textbf{75.} 如果投诉突然升高先做哪三步? & 核实投诉源、收紧策略、复盘通道。 \\
\midrule
\textbf{76.} 如果成功率低先查哪四类原因? & 查号码质量、风控拦截、通道状态、频控限制。 \\
\midrule
\textbf{77.} 如果客户要固定尾号你先问什么? & 问固定尾号/总长/三网一致三要素。 \\
\midrule
\textbf{78.} 如果客户说“没要求”你还要追问什么? & 继续追问码号、回执、QPS、投诉与引流需求。 \\
\midrule
\textbf{79.} 如果大促QPS上万你先拉谁? & 先拉运营和技术协同。 \\
\midrule
\textbf{80.} 如果客户要拉状态你要提醒什么? & 提醒拉取频率、堆积风险和隔离策略。 \\
\midrule
\textbf{81.} 如果客户问“为什么140字不是2条”你怎么答? & 67内1条,超67按67分片,所以140字是3条。 \\
\bottomrule
\end{longtable}

\chapter{F卷:扩展消息类型专题(USSD/二进制短信/闪信)}
\begin{longtable}{P{0.64\textwidth}P{0.3\textwidth}}
\toprule
\mystrong{题目(含选项)} & \mystrong{答案与解释} \\
\midrule
\textbf{1.} 下列关于 USSD 的描述,正确的是:\par \textbf{A.} 典型是“存储转发”\par \textbf{B.} 依赖移动互联网\par \textbf{C.} 属于实时会话型交互\par \textbf{D.} 必须安装App & \ansline{C}\par \expline{USSD是GSM会话型交互协议,强调实时菜单交互,不是短信存储转发。} \\
\midrule
\textbf{2.} USSD 最典型的交互入口是:\par \textbf{A.} 邮件链接\par \textbf{B.} 拨号输入\texttt{*...\#}\par \textbf{C.} 应用内H5\par \textbf{D.} 二维码扫码 & \ansline{B}\par \expline{用户在拨号盘输入特定代码触发USSD会话,这是其经典入口。} \\
\midrule
\textbf{3.} 下列哪项更符合二进制短信(Binary SMS)?\par \textbf{A.} 仅用于文本群发\par \textbf{B.} 负载是二进制数据\par \textbf{C.} 不走短信网络\par \textbf{D.} 不需要终端解析 & \ansline{B}\par \expline{二进制短信的核心是“短信通道承载二进制负载”,常见于控制和配置类场景。} \\
\midrule
\textbf{4.} 二进制短信的典型应用不包括:\par \textbf{A.} 设备参数下发\par \textbf{B.} M2M控制指令\par \textbf{C.} WAP Push\par \textbf{D.} 常规营销文案展示 & \ansline{D}\par \expline{二进制短信偏“控制/配置”用途,常规营销展示通常不选该形态。} \\
\midrule
\textbf{5.} 闪信在技术上属于:\par \textbf{A.} Class 0 SMS\par \textbf{B.} MMS\par \textbf{C.} RCS\par \textbf{D.} 邮件通知 & \ansline{A}\par \expline{Flash SMS 在GSM规范中对应 Class 0。} \\
\midrule
\textbf{6.} 闪信的典型特征是:\par \textbf{A.} 默认长期保存在收件箱\par \textbf{B.} 消息优先弹窗展示\par \textbf{C.} 只能在弱网接收\par \textbf{D.} 仅支持iOS & \ansline{B}\par \expline{闪信强调“强提醒”,通常优先弹出而非常规入箱。} \\
\midrule
\textbf{7.} 下列哪项是闪信在安全侧的主要风险?\par \textbf{A.} 无法显示\par \textbf{B.} 锁屏可见导致信息暴露\par \textbf{C.} 无法计费\par \textbf{D.} 无法送达 & \ansline{B}\par \expline{闪信可能在锁屏界面直接展示内容,存在旁观泄露风险。} \\
\midrule
\textbf{8.} 若业务目标是“功能机环境下实时菜单式查询”,优先建议:\par \textbf{A.} 富媒体短信\par \textbf{B.} 5G消息\par \textbf{C.} USSD\par \textbf{D.} 邮件推送 & \ansline{C}\par \expline{该目标与USSD的能力边界高度匹配。} \\
\bottomrule
\end{longtable}
\appendix
\chapter{修订说明与版本记录}
\section{修订声明}
\begin{riskbox}
本文档已完成可学习化修订,并已执行出版级精修:统一术语、统一客户匿名策略、统一规则版本号。
本版新增“左题右答”双栏结构(A--F卷全覆盖),用于提升做题与核对效率。
\end{riskbox}

\section{版本变更}
\begin{longtable}{P{0.13\textwidth}P{0.16\textwidth}P{0.55\textwidth}}
\toprule
\mystrong{版本} & \mystrong{日期} & \mystrong{变更说明} \\
\midrule
V1.0 & 2026-02-08 & 初版:全知识点题库与答案解析 \\
V2.0 & 2026-02-08 & 出版级精修:术语统一、匿名策略固化、规则版本基线化 \\
V3.0 & 2026-02-08 & 版式升级:A--E卷改为左题右答双栏,去除后置答案重复章节 \\
V3.1 & 2026-02-09 & 新增“扩展消息类型专题(USSD/二进制短信/闪信)”,补全题库覆盖面 \\
\bottomrule
\end{longtable}

\section{出卷与阅卷复用建议}
\begin{keybox}
\begin{enumerate}
  \item 出卷时先锁定规则版本号,避免跨版本混题。
  \item 新增题目时优先复用“客户匿名+行业标签”模板。
  \item 阅卷争议统一回到“出版级作答口径”章节判定。
  \item 每季度更新一次规则基线并抽样重审答案。
\end{enumerate}
\end{keybox}

\section{A.4 对外发布前检查清单}
\begin{keybox}
\begin{enumerate}
  \item 术语是否全部符合“术语统一标准”。
  \item 客户信息是否全部达到 Release-L2 匿名等级。
  \item 规则口径是否全部标注版本号。
  \item 时间窗、计费、回执、频控描述是否与当前规则一致。
  \item 图表标题、单位、缩写(QPS、MO、MT)是否统一。
  \item 是否移除内部群名、个人姓名、私有项目代号。
  \item PDF 元信息与封面版本信息是否一致。
\end{enumerate}
\end{keybox}

\end{document}
